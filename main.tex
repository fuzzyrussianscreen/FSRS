\documentclass[12pt,a4paper]{article}
\usepackage[utf8]{inputenc}
\usepackage[english, russian]{babel }
\usepackage{geometry}
\usepackage{indentfirst}

\geometry{
a4paper,
total={170mm,257mm},
left=20mm,
top=20mm,
}

\titleformat{\normalfont}
\title{Основы научных исследований и наукометрика. Обзорная статья}
\author{Курило Дмитрий ПИмд-11}
\date{}

\begin{document}

\maketitle

\section*{Введение}
Темой исследовательской работы является разработка подхода к анализу аномалий временных рядов основанного на интеллектуальных моделях. На основании этой темы были рассмотрены научные статьи и проведён краткий анализ аналогичных разработок. 
\section*{Обзор статей}
В данной \cite{art1} работе анализируются современные методы обнаружения аномалий и классификации временных рядов. Рассмотрены как статистические методы, так и методы использующие глубокие нейронные сети. Одним из нововведений использование сверточных нейронных сетей в обнаружении аномалий.

В статье \cite{art2} рассмотрены особенности аномалий в данных компьютерных сетей автоматизированных ИС. Предложена обобщённая схема обнаружения сетевых аномалий, а также концепция комплексного подхода к поиску аномалий, объединяющая уже существующие методы.

В работе \cite{art3} рассматриваются нейросетевые методы прогнозирования сетевого трафика с целью прогнозирования зачения объёма пакетов на короткий период. В качестве вывода была предложена трёхслойная нейросетевая модель, решающая поставленную задачу. Другим вариантом является радиально базисные функциональные нейронные сети.

В статье \cite{art4} описывается метод использования целевой функции для обучения нейронной сети методом обратного распространения ошибки на несбалансированных примерах. Данный метод выбран из-за неоднородности исходных данных и показал свою эффективность в результате экспериментов. 

В статье \cite{art5} анализируются наиболее распространённые программные системы управления и диагностики СХД. Рассмотренные программные средства предлагают требуемую функциональность, но не выявляют подробную информацию о видах ошибок и требуют информацию от эксперта предметной области.

В работе \cite{art6} предлагается подход сегментации временных рядов, основанный на сверточных нейронных сетях. Также предложен фреймворк обучения по крупному набору синтетических одномерных данных и последующая настройка по меньшему набору одномерных и многомерных данных с классами аномалий. Для многомерных данных представлена новая сетевая архитектура.

В статье \cite{art7} говорится об использовании генеративно-состязательных сетей машинного обучения для поиска аномалий. Цель исследования выяснить баланс между точностью обнаружения аномалий и частоты получения данных выборки. Обучающая выборка неравномерная и методы учитывают ошибку и частоту пропусков данных. Эксперименты показали работоспособность полученного анализатора при большой буферной зоне, но за счёт увеличения частоты дискретизаций.

В статье \cite{art8} описывается применение рекуррентных нейронных сетей (RNN) с архитектурой долгой краткосрочной памяти (LSTM) и закрытых рекуррентных блоков. Эти блоки помогают пройти ограничения низкочувствительного обнаружения краткосрочных аномалий в исходных данных летательных аппаратов. Результатом экспериментов обнаружение большинства аномалий тестовой выборки и возможность реализации на полётной палубе для обнаружения аномалий в реальном времени.

В статье \cite{art9} рассматривается работа сетей долгой краткосрочной памяти на данных ЭКГ, космического шаттла, потребляемой мощности и данных двигателя. В результате экспериментов с многоуровневой системой были получены данные ошибки прогнозирования. Эти ошибки были промоделированы как многомерное распределение Гаусса.

В статье \cite{art10} исходными данными являются лог файлы систем, позволяющие делать выводы о работе программных и аппаратных технологий. Исследование рассматривает проблемы краткосрочного прогнозирования крупных систем на наборах данных высокой размерности. Из выбранных алгоритмов был выбран наилучший и испытан в режиме реального времени.
\section*{Заключение}

В результате рассмотрения статей разных предметных областей, можно сделать вывод, что нейронные сети являются актуальным инструментом для решения задач множества предметный областей. Особенностью является возможности использования разных структур сетей а так же настройка различных параметров, позволяющие учитывать особенности входных данных. 

\begin{thebibliography}{10}

\bibitem{art1} Автоматический поиск аномалий во временных рядах. Соболев К. В. 2018.

\bibitem{art2} Анализ специфики и проблематики процессов поиска аномалий в сетевых данных. Линдгридин А. Н. 2021.

\bibitem{art3} Выявление аномалий в сетевом трафике на основе нейросетевого моделирования динамики изменения объёмов IP-пакетов. Гасанов В.И. 2018.

\bibitem{art4} Обнаружение аномалий сетевого трафика методом глубокого обучения. Зуев В.Н. 2021.

\bibitem{art5} Обзор подходов к обнаружению сбоев в системах хранения данных. Успенский М. Б. 2019.

\bibitem{art6} Time Series Anomaly Detection Using Convolutional Neural Networks and Transfer Learning. Tailai Wen, Roy Keyes. 2019.

\bibitem{art7} Anomaly Detection and Sampling Cost Control via Hierarchical GANs, Chen Zhong, M. Cenk Gursoy, Senem Velipasalar, 2020.

\bibitem{art8} Anomaly detection in aircraft data using recurrent neural networks, Anvardh Nanduri, Lance Sherry, 2019.

\bibitem{art9} Long Short Term Memory Networks for Anomaly Detection in Time Series, Pankaj Mallustra, Lovrikosh Vis, Gautam Agarwal, 2017.

\bibitem{art10} Logs analysis to search for anomalies in the functioning of large technology platforms, Maxim Dunaev, Konstantin Zaytsev, 2019.


\end{thebibliography}

\end{document}
