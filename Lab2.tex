\documentclass[aspectratio=1610,17pt,utf8]{beamer}


\usepackage[T1]{fontenc}
\usepackage[utf8]{inputenc}
\usepackage[russian]{babel}
\usepackage{graphicx} % graphics
\usepackage{mathabx}
\usepackage{mathpazo}
\usepackage{eulervm}

\usetheme{unipassau}


% title slide definition
\title[Shorter Title]{Разработка подхода к анализу аномалий временных рядов, основанного на интеллектуальных моделях}

\author{\textbf{Курило Дмитрий, ПИмд-11}}
\institute[]
{
  Ульяновский государственный технический университет
}
\date{декабрь 2021}

%--------------------------------------------------------------------
%                            Titlepage
%--------------------------------------------------------------------

\begin{document}

	\maketitle
	
	\begin{frame}
		\frametitle{Цель исследования}
		\begin{quotation}
			Целью работы является разработка подхода к анализу аномалий временных рядов, основанного на интеллектуальных моделях и их прототипирование.
		\end{quotation}
	\end{frame}
	
	
	\begin{frame}
		\frametitle{Объект и предмет исследования}
		\begin{quotation}
			Объектом исследования являются временные ряды с несколькими показателями, взятые с реальных физических объектов
		\end{quotation}
		
		\begin{quotation}
			Предметом исследования являются аномалии временных рядов и последующий логический вывод на основании базы знаний.
		\end{quotation}
	\end{frame}

	\begin{frame}
		\frametitle{Задачи}
		Достижение поставленной цели предполагает реализацию следующих задач:
		\begin{itemize}
		    \item Анализ аномалий временных рядов характеристик в диагностических данных
			\item Формализация и прототипирование метода обобщения результатов анализа аномалий и закономерностей изменения с использованием онтологической модели
			
		\end{itemize}
	\end{frame}	
	
	\begin{frame}
		\frametitle{Задачи}
		Достижение поставленной цели предполагает реализацию следующих задач:
		\begin{itemize}
		   \item Обучение статической и динамической составляющих базы знаний для актуализации интерпретации характерных ей коллизий
			\item Проведение экспериментов с полученными моделями и с применением базовых методов машинного обучения
		\end{itemize}
	\end{frame}
	
	\begin{frame}
		\frametitle{Научная новизна}
		\begin{quotation}
			Научная новизна работы состоит в гибридизации алгоритмов интеллектуального детектирование аномалий четких временных рядов и нечетких правил баз знаний для логического вывода.
		\end{quotation}
	\end{frame}
	

	\begin{frame} 
	    \begin{quotation}
			Исходными данными системы являются журналы девяти газовых скважин с пятью измерениями датчиков скважин и двумя геологическими измерениями.
		\end{quotation}
		\frametitle{Исходные данные системы}
		%\begin{figure}
        
        %\caption{Данные}
		%\end{figure}
	\end{frame}
	

	
	\begin{frame} 
		\frametitle{Положение выносимые на защиту}
		\begin{quotation}
		    Программный модуль для извлечения и детектирования аномалий временных рядов показателей, извлеченных из логов.
		    
		    Программный модуль логического вывода на основании синтеза базы знаний системы и результатов поиска аномалий.
	    \end{quotation}
	\end{frame}
	
	\begin{frame} 
		\frametitle{Обзор научных работ по схожей тематике}
		    \begin{itemize}
		   \item Автоматический поиск аномалий во временных рядах. Соболев К. В. 2018.
		   \item Обзор подходов к обнаружению сбоев в системах хранения данных. Успенский М. Б. 2019.
		   \item Time Series Anomaly Detection Using Convolutional Neural Networks and Transfer Learning. Tailai Wen, Roy Keyes. 2019.
		\end{itemize}
	\end{frame}


    \begin{frame} 
		\frametitle{Спасибо за внимание}
		   \begin{quotation}
		    Курило Дмитрий
		    
		    dimokkurilo@gmail.com
	    \end{quotation}
	\end{frame}
%%%%%%%%%%%

\end{document}
